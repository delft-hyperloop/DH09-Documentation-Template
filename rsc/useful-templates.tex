\makeatletter
\newcounter{usefultemplate}
\newcommand\@defusefultemplate[1]{
    \@namedef{usefultemplatetitle\theusefultemplate}{#1}
    \stepcounter{usefultemplate}
}
\newcommand\@renderusefultemplates{
    \@makecount{renderusefultemplatescnt}
    \whiledo{\value{renderusefultemplatescnt}<\value{usefultemplate}}{
        \ifthenelse{\value{renderusefultemplatescnt}=0}{}{\newpage}
        \section*{\@nameuse{usefultemplatetitle\therenderusefultemplatescnt}}
        \addcontentsline{toc}{section}{\@nameuse{usefultemplatetitle\therenderusefultemplatescnt}}
        \@codeandoutput{__useful_templates-\therenderusefultemplatescnt}{useful-templates-\therenderusefultemplatescnt}{Code listing for \@nameuse{usefultemplatetitle\therenderusefultemplatescnt}}
        \stepcounter{renderusefultemplatescnt}
    }
}

\begin{filecontents*}{__useful_templates-\theusefultemplate.tex}
\begin{figure}[tph]
    \centering
    \includegraphics[width=0.5\linewidth]{example-image}
    \caption{Template of a Figure}
    \label{fig:descriptive-label}
\end{figure}
\end{filecontents*}
\@defusefultemplate{Figure}

\begin{filecontents*}{__useful_templates-\theusefultemplate.tex}
\begin{figure}[tph]
    \centering
    \begin{subfigure}{0.3\textwidth}
    \centering
    % Don't adjust this width
    \includegraphics[width=\textwidth]{example-image-a}
    % Caption for the subfigure
    \caption{Caption A.}
    % Label for the subfigure
    \label{subfig:subfig-example-a}
    \end{subfigure}
    \hfill
    \begin{subfigure}{0.3\textwidth}
    \centering
    \includegraphics[width=\textwidth]{example-image-b}
    \caption{Caption B.}
    \label{subfig:subfig-example-b}
    \end{subfigure}
    \hfill 
% Caption for the figure as a whole
\caption{Caption for all sub-figures. Also works with tables (use subtable)}
% Label for the figure as a whole
\label{fig:subfig-example}
\end{figure}
\end{filecontents*}
\@defusefultemplate{2 Subfigures (works with both images and tables)}

\begin{filecontents*}{__useful_templates-\theusefultemplate.tex}
\begin{figure}[tph]
    \centering
    \begin{subfigure}{0.3\textwidth}
    \centering
    \includegraphics[width=\textwidth]{example-image-a} % Don't adjust this width
    \caption{Caption A.}            % Caption for the subfigure
    \label{subfig:subfig-example-a} % Label for the subfigure
    \end{subfigure}
    \hfill
    \begin{subfigure}{0.3\textwidth}
    \centering
    \includegraphics[width=\textwidth]{example-image-b}
    \caption{Caption B.}
    \label{subfig:subfig-example-b}
    \end{subfigure}
    \hfill 
    \begin{subtable}[b]{0.3\textwidth}
        \begin{tabular}{L{0.1\linewidth} C{0.4\linewidth} r}
        \hhline{===} 
        \textbf{A} & \textbf{B} & \textbf{C}   \\ \hline 
        D & E & F   \\
        G & H & I   \\
        \hhline{===}
        \end{tabular}
        \caption{Caption C}
      \label{table:table}
  \end{subtable}
\caption{Caption for all sub-figures.}  % Caption for the figure as a whole
\label{fig:subfig-example}              % Label for the figure as a whole
\end{figure}
\end{filecontents*}
\@defusefultemplate{3 Subfigures (can be images or tables)}

\begin{filecontents*}{__useful_templates-\theusefultemplate.tex}
\begin{wrapfigure}{r}{0.4\linewidth}
    \centering
    \includegraphics[width=0.4\linewidth]{example-image}
    \caption{Template of a Wrap Figure}
    \label{fig:descriptive-label}
\end{wrapfigure}

\lipsum[1]
\end{filecontents*}

\@defusefultemplate{Wrapfigure (text wraps around it)}

\begin{filecontents*}{__useful_templates-\theusefultemplate.tex}
\end{filecontents*}
\@defusefultemplate{Table}

\begin{filecontents*}{__useful_templates-\theusefultemplate.tex}
\end{filecontents*}
\@defusefultemplate{A3 (vertical)}


\begin{filecontents*}{__useful_templates-\theusefultemplate.tex}
\end{filecontents*}
\@defusefultemplate{A3 (horizontal)}

\begin{filecontents*}{__useful_templates-\theusefultemplate.tex}
\end{filecontents*}
\@defusefultemplate{2 columns}

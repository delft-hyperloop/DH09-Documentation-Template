\chapter*{Consistency Guidelines}
\label{chapter:template-guidelines}
These guidelines aim to develop consistency in report layout, use of English, and use of \LaTeX. For any questions, remarks or suggestions on the guidelines, please contact Dinu/Pablo.

\section*{\LaTeX \ guidelines}
\begin{enumerate}[label=LTX.\arabic*]
    \item Never write \verb|&| directly in your text, always write \verb|\&|. The \& symbol is reserved for other stuff
    \item Use \verb|\autoref{label-here}| to reference elements. Do \textbf{not} use \verb|\ref{label-here}| (autoref automatically writes "Figure" or "Table" in the correct format).
\end{enumerate}
\section*{On General Consistency}
\label{sec:general-consistency}
\begin{enumerate}[label=G.\arabic*]
    \item Use American English for documenting/reporting. 'z' not 's'!
    \item When referring to ourselves, say "the team" or "DH09", not "me", "us", etc.
    \item Write everything in impersonal English ("this was analyzed" rather than "we analyzed this"). It's kind of annoying to write but much more professional.
    \item Do not use abbreviations for verbs, write out in full 'is not' or 'cannot'.
    \item End sentences with a dot, even in figures, lists, captions, and tables. Chapter and section titles are the exception.
    \item Do not use 'he/she'. Try to use alternatives or use 'they' if appropriate.
    \item Each identifier (e.g. for a requirement) should be unique. %You can make use of \verb|[label=ABC\arabic*]|.
    \item Capitalize words in section and chapter titles, except articles, prepositions, and conjunctions (the, and, or, for, etc.). The first word is always capitalized.
    \item You can use \verb|\restoregeometry| to fix layout after inserting an A3 page.
    \item If colors have meaning, add color-coding (i.e. legend) for colorblind accessibility.
    \item Use \verb|\,| to get a space in large numbers, like 100\,000 (instead of 100 000) (it's stupid i know but it looks nice).
    \item Add abbreviations to the nomenclature table. \unsure{do we want nomenclature?}
    \item Paragraphs should be ended with a \textbf{??????????????} \unsure{how are we doing that}
\end{enumerate}
\section*{On Label Consistency} \label{sec:label-consistency}
\begin{enumerate}[label=L.\arabic*]
    \item Set a unique and \textbf{\textit{describing}} label for each new element (chapters, figures, tables...). Generally, chapters/section labels should be the same as the chapter/section title and figures/tables should describe their contents in a few words (2-3 is good).
    \item DO NOT USE SPACES OR '\&' IN YOUR LABELS. SEPARATE WORDS BY '-'. 
    \item The standard word to be used at the start of the label for different types of elements is: 
    \begin{enumerate}[label*=.\arabic*]
        \item \textit{chapter:} for chapter.
        \item \textit{appendix:} for appendix chapter.
        \item \textit{sec:} for section.
        \item \textit{subsec:} for subsection.
        \item \textit{tab:} for table.
        \item \textit{fig:} for figure.
        \item \textit{subfig:} for subfigure.
        \item \textit{note:} for a footnote. Labels for footnotes are optional.
    \end{enumerate}
    
    \item You can also use \verb|\nameref{label-here}| to reference items by their name, as opposed to their number, if it adds clarity.
\end{enumerate}

\section*{On Math Notation Consistency}
\label{sec:math-consistency}
\begin{enumerate}[label=M.\arabic*]
    \item USE \verb|'\cdot'| (result: $\cdot$), NOT '*' OR '·'. IF YOU DON'T I WILL PERSONALLY CHASE YOU DOWN. It is also allowed to use \verb|\times| (result: $\times$) if needed, but it isn't the standard.
    \item Use \$ \$ or \verb|\( \)|for math notation in text.
    \item Use \verb|\begin{equation} \end{equation}| for equations/longer math notation.
    \item Units should not have square brackets, but instead use normal text (and any math text if necessary, e.g. squared units). Correct: 5 m\(^2\). Incorrect: 5 [$m^2$].
    \item Use \verb|\text{}| if units or other text should appear in an equation. E.g. $ \vec{\dot{x}} = \text{A} \vec{x} + \text{B} \vec{u} + \text{some txt} \cdot 1000$
    \item For short units, you can use slash, m/s. For longer units, use negative powers, kg m\(^2\)s\(^{-2}\).
    \item Add symbols to the nomenclature. \unsure{nomenclature again}
\end{enumerate}

\section*{On Table Consistency}
\label{sec:table-consistency}
\begin{enumerate}[label=T.\arabic*]
    \item Start and end the table body with a double rule using \verb|\hhline{===}|. The number of '\verb|=|' corresponds to the number of columns in the table. See \autoref{tab:example} as an example.
    
    \item For tables, the caption goes above the body (which means the \verb|\caption| command goes before the \verb|\begin{tabular}| command).
    \item Table headers (first row) should be in boldface, use \verb|\textbf{}|.
    \item The best way to make tables easily is to use a latex table generator (I use \href{https://www.tablesgenerator.com/}{{blue}{{this page}}}).
    \item When defining columns (next to \verb|\begin{tabular}|), use \verb|l|, \verb|c|, or \verb|r| for left, center, or right text alignment of the column. The width of the column is automatically set by \LaTeX. See the source code of \autoref{tab:example} for an example.
    \item If you want to have custom width for the columns, use \verb|L/R/C{0.1\linewidth}| (choose a letter depending on text alignment) in the definition of your columns. Replace 0.1 with whatever percentage of the linewidth you want to set. See the source code of \autoref{tab:example} for an example.
    \item Put \verb|[tph]| after \verb|\begin{table}| to make the table appear exactly where you placed it and not somewhere random.
\end{enumerate}
\begin{table}[tph]
    \centering
    \caption{This is my caption.} % Caption on top of \begin{tabular}
    \label{tab:example}
    \begin{tabular}{L{0.1\linewidth} C{0.4\linewidth} r} % 1st column: 10% of line width, text aligned left; 2nd column: 40% of line width, text is centered; 3rd column: automatic width, text aligned right
        \hhline{===} % Double line, three columns -> three = signs
        \textbf{A} & \textbf{B} & \textbf{C}   \\ \hline % Bold table headers, \hline after headers
        D & E & F   \\
        G & H & I   \\
        \hhline{===}
    \end{tabular}
\end{table}

\section*{On Figure Consistency}
\label{sec:figure-consistency}
\begin{enumerate}[label=F.\arabic*]
    \item For figures, the caption goes under the figure.
    \item Use SI-units on the axis of graphs, unless other units (e.g. km/h) add value/clarity to the graph.
    \item To resize an image use \verb|\includegraphics[width=X\textwidth]{img.png}|, where X is the percentage of text width you want the image to take up.
    \item If you know a figure needs to be there but you don't have it yet, type \verb|example-image|, \verb|example-image-a|, \verb|example-image-b|, or \verb|example-image-c| for a placeholder (see \autoref{fig:example-image} for an example).
    \item For side-by-side figures, use the \verb|\begin{subfigure}| command. Copy-paste the template from \autoref{fig:subfig-example}.
\end{enumerate}

\begin{figure}[tph]
    \centering
    \includegraphics[width=0.5\linewidth]{example-image} % Image takes 50% of text, but it can be adjusted
    \caption{Caption for figure.}
    \label{fig:example-image}
\end{figure}

\begin{figure}[h]
    \centering
    \begin{subfigure}{0.3\textwidth} % The subfigure takes 30% of the text
    \centering
    \includegraphics[width=\textwidth]{example-image-a} % Don't adjust this width
    \caption{Caption A.}            % Caption for the subfigure
    \label{subfig:subfig-example-a} % Label for the subfigure
    \end{subfigure}
    % add an empty line to create a new row of images

    \begin{subfigure}{0.2\textwidth}
    \centering
    \includegraphics[width=\textwidth]{example-image-a}
    \caption{Caption A.}
    \label{subfig:subfig-example-a-repeated}
    \end{subfigure}
    \hfill % add \hfill to distribute the images evenly across the width of the page
    \begin{subfigure}{0.2\textwidth}
    \centering
    \includegraphics[width=\textwidth]{example-image-b}
    \caption{Caption B.}
    \label{subfig:subfig-example-b}
    \end{subfigure}
    \hfill 
    \begin{subfigure}{0.2\textwidth}
    \centering
    \includegraphics[width=\textwidth]{example-image-c}
    \caption{Caption C.}
    \label{subfig:subfig-example-c}
    \end{subfigure}

    % ... (add more if needed).
    \caption{Caption for all sub-figures.}  % Caption for the figure as a whole
    \label{fig:subfig-example}              % Label for the figure as a whole
\end{figure}
\nameref{fig:subfig-example}
\section*{On Use of Sources}
\begin{enumerate}[label=SO.\arabic*]
    \item Follow the IEEE guidelines.
    \item Example of a bibtex citation \cite{WinNT}: (find more documentation \href{https://bibtex.eu/types/}{\color{blue}{\underline{here}}} and \href{https://www.overleaf.com/learn/latex/Bibliography_management_with_bibtex}{\color{blue}{\underline{here}}})

\begin{verbatim}
@techreport{FDD_DH08, 
 author = {DH08}, 
 title = {Final Design Document}, 
 month = "December", 
 year  = "2023", 
 type  = "Technical Report", 
}
\end{verbatim}
    \item For citing webpages use the \verb|@online| type:
    \begin{verbatim}
@online{WinNT,
  author = {MultiMedia LLC},
  title = {{MS Windows NT} Kernel Description},
  year = 1999,
  url = {http://web.archive.org/web/20080207010024/http://www.808multimedia.com/winnt/kernel.htm},
  urldate = {2010-09-30}
}
\end{verbatim}
\end{enumerate}
\section*{On department-specific sources in the template} \label{sec:deptartment-specific-sources}

Every single department has its own file under \textit{rsc/depts/}, which they can modify as they see fit. In order to prevent incompatibilities with the template itself and/or other departments, please stick to the following guidelines:

\begin{enumerate}[label=DSS.\arabic*]
    \item You can change the list of names in the respective department file. 
    \item All exported commands and environments should start with the name of the department. For example: \verb|\newcommand{\levitationtable}|, \verb|\newcommand{\powertrainpin}|. Unambiguous abbreviations of the departments are also allowed here, instead of the full name.
    \item All commands used internally have the naming requirements from DSS.1. It is allowed to use \verb|\makeatletter| and have those commands used internally start with the \textit{at} character (\textit{@}), before the department name. In the examples above: \verb|\@levitationtable| or \verb|\@powertrainpin|.
\end{enumerate}
